\documentclass[8pt]{beamer}

\usepackage{minted}
\newminted{scala}{linenos,bgcolor=dbg,frame=single}
\newminted{text}{linenos,bgcolor=lbg,frame=single}
\usemintedstyle{monokai}
\newcommand{\codeclass}[1]{\texttt{\textcolor[rgb]{0.4,0.85,0.9375}{#1}}}
\newcommand{\codemethod}[1]{\texttt{\textcolor[rgb]{0.457,0.4375,0.3667}{#1}}}
\newcommand{\codeobject}[1]{\texttt{\textcolor[rgb]{0.65,0.885,0.18}{#1}}}
\newcommand{\codevalue}[1]{\texttt{\textcolor[rgb]{0.975,0.15,0.45}{#1}}}
\newcommand{\codenumber}[1]{\texttt{\textcolor[rgb]{0.68,0.5,1}{#1}}}
\newcommand{\codestring}[1]{\texttt{\textcolor[rgb]{0.9,0.86,0.45}{#1}}}

\begin{document}
  \definecolor{dbg}{rgb}{0.25,0.25,0.25}
  \definecolor{lbg}{rgb}{0.75,0.75,0.75}
  \begin{center}
    \textsc{\huge{Introduction to Rx Observables}}
  \end{center}
  first: let's define some convenience methods and values (will use it quite often...):
  \begin{scalacode}
import rx.{Observer => JObserver, Subscription => JSubscription}  
  
def startOnThread(body: => Unit): Thread = {
  val t = new Thread {
    override def run = body
  }
  t.start
  t
}

val zeroTime = System.currentTimeMillis

def threadTimeString = {
  val name = Thread.currentThread.getName
  val time = System.currentTimeMillis - zeroTime
  s"current time on $name is: $time."
}
  \end{scalacode}
  \newpage
  
  a ``simple'' example implementation for an \codeclass{Observable} serving as a tick source:
  \begin{scalacode}
val ticks: Observable[String] = Observable(observer => {
  var cancelled = false
  val t = startOnThread {
    try {
      while (!cancelled) {
	try {
	  observer.onNext(threadTimeString)
	  Thread.sleep(100)
	} catch {
	  case e: java.lang.InterruptedException => //DO NOTHING
	}
      }
      observer.onCompleted()
    } catch {
      case e: Throwable => observer.onError(e)
    }
  }
  Subscription(new JSubscription {
    override def unsubscribe() = {
      cancelled = true
      t.interrupt
      t.join
    }
  })
})
  \end{scalacode}
  \newpage
  well, how would we use it? simple:
  \begin{scalacode}
val subscription = ticks.subscribe(println(_))
Thread.sleep(1000)
subscription.unsubscribe
  \end{scalacode}
  \medskip
  
  which would print something like:
  \begin{textcode}
current time on Thread-0 is: 222.
current time on Thread-0 is: 337.
current time on Thread-0 is: 437.
current time on Thread-0 is: 538.
current time on Thread-0 is: 638.
current time on Thread-0 is: 739.
current time on Thread-0 is: 839.
current time on Thread-0 is: 940.
current time on Thread-0 is: 1040.
current time on Thread-0 is: 1141.
  \end{textcode}
  \newpage
  
  but if want better control, in case we want to treat errors or execute something when 
  the \codeclass{Ovservable} announces completion, we will need to define an \codeclass{Observer}
  \begin{scalacode}
val ticksConsumer: Observer[String] = Observer(
  new JObserver[String]{
    def onNext(s: String) = println(s)
    def onError(e: Throwable) = logger.error(e.getMessage)
    def onCompleted() = println("DONE!")
  }
)
  \end{scalacode}
  \smallskip
  
  usage is similar:
  \begin{scalacode}
val subscription = ticks.subscribe(ticksConsumer)
Thread.sleep(1000)
subscription.unsubscribe
  \end{scalacode}
  
  \newpage
  
  and now, the output contains a ``DONE!'' print:
  \begin{textcode}
current time on Thread-0 is: 201.
current time on Thread-0 is: 314.
current time on Thread-0 is: 414.
current time on Thread-0 is: 515.
current time on Thread-0 is: 615.
current time on Thread-0 is: 716.
current time on Thread-0 is: 816.
current time on Thread-0 is: 917.
current time on Thread-0 is: 1017.
current time on Thread-0 is: 1118.
DONE!   
  \end{textcode}
  \medskip
  
  but the shown code has some problems:
  \begin{itemize}
    \item \text{the \codeclass{Observable}'s code is somewhat boilerplate}
    \item \text{the whole chain of execution is done on a single thread}
    \item \text{\codeclass{Scheduler}s...?}
    \item \text{what would happen if more than one \codeclass{Observer} subscribes to this \codeclass{Observable}?} %answer: a new thread will be created for every observer subscribing.
  \end{itemize}
  
  \newpage
  let's try to improve our solution. \\ 
  first, consider a very naive implementation of a ``ticks source'':
  \begin{scalacode}
object TickSource {
  private var callbacks: Map[Int, () => Unit] = Map.empty
  private var counter = 0
  private var cancelled = false
  private val t = startOnThread {
    while (!cancelled) {
      Thread.sleep(100)
      callbacks.foreach(_._2())
    }
  }

  def onTick(callback: => Unit): Int = {
    counter = counter + 1
    callbacks = callbacks +(counter, () => callback)
    counter
  }

  def remove(key: Int): Unit = {callbacks = callbacks - key}

  def shutdown = {cancelled = true; t.join}
}
  \end{scalacode}
  \smallskip
    
  \texbf{NOTE:} the above code is very naive. I kept it short for this toy example.
  a real implementation though must take care of synchronizing the \codemethod{onTick} and \codemethod{remove} methods (among other things...).
  \newpage
  
  also, we'll need to define the \codeclass{Observable} that will use \codeobject{TickSource} as it's source.
  \begin{scalacode}
val ticks: Observable[String] = Observable(observer => {
  val key = TickSource.onTick {
    observer.onNext(threadTimeString)
  }
  Subscription(new JSubscription {
    override def unsubscribe() = TickSource.remove(key)
  })
})
  \end{scalacode}

  so, now we can use it exactly as before:
  \begin{scalacode}
val subscription = ticks.subscribe(ticksConsumer)
Thread.sleep(1000)
subscription.unsubscribe
TickSource.shutdown //just to ensure the proccess won't hang...
  \end{scalacode}
  \smallskip
  
  \textbf{NOTE:} there is no \codemethod{onCompleted} method in this scenario. the source is infinite, and does not ``completes''
  \newpage
  
  and we'll get:
  \begin{textcode}
current time on Thread-0 is: 271.
current time on Thread-0 is: 385.
current time on Thread-0 is: 485.
current time on Thread-0 is: 586.
current time on Thread-0 is: 686.
current time on Thread-0 is: 787.
current time on Thread-0 is: 887.
current time on Thread-0 is: 988.
current time on Thread-0 is: 1088.
current time on Thread-0 is: 1189.
  \end{textcode}
  \smallskip
  
  well, that took care of some problems. but it's still not perfect.
  we would want to sepparate the source work and consume work to be done on different threads.
  luckily, there's a simple to use API for that: \codemethod{observeOn}.
  \begin{scalacode}
val scheduler = rx.lang.scala.concurrency.Schedulers.newThread
val subscription = ticks.observeOn(scheduler).subscribe{s =>
  println(s"$s\n${threadTimeString}")
}
Thread.sleep(1000)
subscription.unsubscribe
  \end{scalacode}
  
  \newpage
  and the output now looks like:
  \begin{textcode}
current time on Thread-0 is: 305.
current time on RxNewThreadScheduler-1 is: 328.
current time on Thread-0 is: 426.
current time on RxNewThreadScheduler-1 is: 427.
current time on Thread-0 is: 527.
current time on RxNewThreadScheduler-1 is: 527.
current time on Thread-0 is: 627.
current time on RxNewThreadScheduler-1 is: 628.
current time on Thread-0 is: 728.
current time on RxNewThreadScheduler-1 is: 729.
current time on Thread-0 is: 829.
current time on RxNewThreadScheduler-1 is: 829.
current time on Thread-0 is: 929.
current time on RxNewThreadScheduler-1 is: 930.
current time on Thread-0 is: 1030.
current time on RxNewThreadScheduler-1 is: 1030.
current time on Thread-0 is: 1130.
current time on RxNewThreadScheduler-1 is: 1131.   
  \end{textcode}

  since a ``NewThread'' \codeclass{Schduler} is usually not what we want, \\
  when in doubt, you can use this \codeclass{Schduler} instead:
  \begin{scalacode}
import rx.lang.scala.concurrency.Schedulers.executor
import scala.concurrent.ExecutionContext.Implicits.global
val scheduler = executor(global)
  \end{scalacode}
  \newpage
  
  cool stuff:
  \begin{scalacode}
def actOn(ss: Seq[String]) = {
  val t = threadTimeString
  val s = ss.mkString("\n")
  println(s"$t\n${t.map(_ => '*')}\n$s\n")
}

val multipleTicksConsumer = Observer(new JObserver[Seq[String]]{
  def onNext(ss: Seq[String]) = actOn(ss)
  def onError(e: Throwable) = logger.error(e.getMessage)
  def onCompleted() = println("DONE!")
})

val scheduler2 = rx.lang.scala.concurrency.Schedulers.newThread

//previous ticks was a toy example. there's a method call for that:
val ticks1 = Observable.interval(100 millis, scheduler)
val ticks2 = ticks1.map(_ => threadTimeString)
val ticks3 = ticks2.observeOn(scheduler2)
val ticks4 = ticks3.buffer(500 milliseconds, 4)

val subscription = ticks4.subscribe(multipleTicksConsumer)
Thread.sleep(1000)
subscription.unsubscribe
  \end{scalacode}
  \newpage
  
  and the output:
  \begin{textcode}
current time on RxNewThreadScheduler-1 is: 611.
***********************************************
current time on ForkJoinPool-1-worker-5 is: 274.
current time on ForkJoinPool-1-worker-5 is: 386.
current time on ForkJoinPool-1-worker-3 is: 487.
current time on ForkJoinPool-1-worker-5 is: 587.

current time on RxNewThreadScheduler-1 is: 991.
***********************************************
current time on ForkJoinPool-1-worker-3 is: 688.
current time on ForkJoinPool-1-worker-5 is: 789.
current time on ForkJoinPool-1-worker-3 is: 889.
current time on ForkJoinPool-1-worker-5 is: 990.
  \end{textcode}
  \medskip
  
  \begin{center}
    \begin{tabular}{ | c | c | c | }
      \hline
      name & type & runs on \\ \hline
      ticks1 & Observable[Long] & ForkJoinPool-1-worker-N \\ \hline
      ticks2 & Observable[String] & ForkJoinPool-1-worker-N \\ \hline
      ticks3 & Observable[String] & RxNewThreadScheduler-1 \\ \hline
      ticks4 & Observable[Seq[String]] & RxNewThreadScheduler-1 \\ \hline
    \end{tabular}
  \end{center}
  \newpage
  
  \begin{center}
    \textsc{\huge{Appendix}}
  \end{center}  
  
  How to write an \codeclass{Observable} the ``scala way'' with no \codeobject{TickSource},\\
  and a ``built in'' \codeclass{Scheduler}:
  \begin{scalacode}
val ticks: Observable[String] = Observable(observer => {
  scheduler.scheduleRec(self => {
    observer.onNext(threadTimeString)
    Thread.sleep(100)
    self
  })
})
  \end{scalacode}
  \smallskip
\end{document}
